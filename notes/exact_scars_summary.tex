
\documentclass[11pt]{article}
\usepackage{amsmath,amsfonts,graphicx}
\usepackage{subcaption}  % For subfigures
\usepackage{float}       % For the [H] placement specifier
\usepackage{authblk}
\usepackage{geometry}
\geometry{margin=1in}
\usepackage{hyperref}
\usepackage{xcolor}


\title{Summary of Recent Research on Exact Quantum Many-Body Scars}
\author{Camilla Polvara}
\date{\today}

\begin{document}

\maketitle

\section*{Overview}

This document summarizes recent research focused on \textbf{exact quantum many-body scars} in spin systems, where analytic expressions for the scarred eigenstates are known. These studies circumvent the need to analyze the full Hamiltonian spectrum, instead targeting entanglement properties and reduced density matrices (RDMs) of analytically understood scarred states. All models are assumed to be in 1-dimension and to have periodic boundary conditions (PBC), unless otherwise specified.

\vspace{0.5cm}

\section*{1. Dimer State Scar}
\begin{itemize}
    \item \textbf{Model:} Spin-1/2 MG-like model featuring an exact dimer scar state.\\
    This is a one-dimensional spin-$\tfrac{1}{2}$ model with two- and three-body interactions. The Hamiltonian of the model depends on a parameter $t \in \mathbb{R}$ and is given by
    
    \begin{equation}
    H(t) = H_{\mathrm{MG}} + t\, C_{\mathrm{SC}},
    \end{equation}
    where
	\begin{equation}
	H_{\mathrm{MG}} = \sum_{j=1}^{L} \left[ \left( \mathbf{S}_j + \mathbf{S}_{j+1} + \mathbf{S}_{j+2} \right)^2 - \frac{3}{4} \right],
	\label{eq:HMG}
	\end{equation}
	\begin{equation}
	C_{\mathrm{SC}} = \sum_{j=1}^{L} \mathbf{S}_j \cdot \left( \mathbf{S}_{j+1} \times \mathbf{S}_{j+2} \right),
	\label{eq:CSC}
	\end{equation}
	and $\mathbf{S}_j = \left(S^x_j, S^y_j, S^z_j\right)$ is the spin-$\tfrac{1}{2}$ operator acting on site $j$:  
	\begin{align}
	S^x_j &= \frac{1}{2}
	\begin{pmatrix}
	0 & 1 \\
	1 & 0
	\end{pmatrix}_j, \\
	S^y_j &= \frac{1}{2}
	\begin{pmatrix}
	0 & -i \\
	i & \phantom{-}0
	\end{pmatrix}_j, \\
	S^z_j &= \frac{1}{2}
	\begin{pmatrix}
	1 & 0 \\
	0 & -1
	\end{pmatrix}_j.
	\end{align}
	
	The first term $H_{\text{MG}}$ is the Hamiltonian of the Majumdar-Ghosh model (which has the dimer states as ground states), while the second term $C_{\text{SC}}$ is the scalar spin chirality.\\ 
	

    
     There is just one scarred state, which is the ground state of the Majumdar-Ghosh Hamiltonian, and it's given by the singlet state:
    
    \begin{equation}
    | \text{dimer} \rangle = \left( 2 + \left( -\frac{1}{2} \right)^{\frac{L}{2} - 2} \right)^{-\frac{1}{2}} \left( |\Psi_1\rangle + |\Psi_2\rangle \right).
    \end{equation}

    where $L$ is the (even) number of sites, and
    
    \begin{align}
    &|\Psi_1\rangle = | \text{sing} \rangle_{1,2} \otimes | \text{sing} \rangle_{3,4} \otimes \cdots \otimes | \text{sing} \rangle_{L-1,L}, \\
    &|\Psi_2\rangle = | \text{sing} \rangle_{2,3} \otimes | \text{sing} \rangle_{4,5} \otimes \cdots \otimes | \text{sing} \rangle_{L,1},
    \end{align}
    
    are the two dimer states, with
    
    \begin{equation}
    | \text{sing} \rangle_{i,j} = \frac{1}{\sqrt{2}} \left( | \uparrow \downarrow \rangle_{i,j} - | \downarrow \uparrow \rangle_{i,j} \right).
    \end{equation}
    
    $| \text{sing} \rangle_{i,j}$ the normalized spin singlet between sites $i,j$.\\
    The scarred state is annihilated by $C_{\text{SC}}$ (as well as by $H_{\text{MG}}$), and thus it is a zero-energy eigenstate of $H$: $H(t)| \text{dimer} \rangle = 0$.
       
    \item \textbf{Reference:} \href{https://journals.aps.org/prb/abstract/10.1103/PhysRevB.108.155102}{Phys. Rev. B 108, 155102 (2023)}.
    \item \textbf{Analysis:}
    \begin{itemize}
        \item Verified the bipartite entanglement entropy against the analytic prediction from the paper.\\ For $L=14,16,18$:
        
        \begin{equation}
        S_A \approx 1.386 \approx 2 \ln 2
        \end{equation}
        
        Because $S_A$ evaluates to $\approx 2 \ln 2$ for increasing $L$, the dimer scar satisfy the \textit{Area Law} for entanglement: in one dimension, the entanglement entropy remains constant with increasing system size.
        
        \item Computed the ranks of all 2-site, 3-site, and 4-site adjacent reduced density matrices:
        
	\begin{table}[h]
	\centering
	\begin{tabular}{|c|ccc|}
	\hline
	\textbf{Dimer scar} & \multicolumn{3}{c|}{\textbf{RDM  Rank}} \\
	\cline{2-4}
	& \textbf{2-sites} & \textbf{3-sites} & \textbf{4-sites} \\
	\hline
	\text{$| \text{dimer} \rangle$} & 4 & 4 & 5 \\
	\hline
	\end{tabular}
	\caption{RDM ranks for 2, 3, and 4 adjacent sites in the dimer scar state for $L=18$.}
	\label{tab:ranks}
	\end{table}

\end{itemize}
\end{itemize}

\vspace{0.3cm}

\section*{2. Domain-Wall Conserving Spin-1/2 Model}
\begin{itemize}
    \item \textbf{Model:} Spin-1/2 model with domain-wall conserving dynamics.\\
    
     For a system $L$ spins-$\tfrac{1}{2}$ on a chain, the Hamiltonian reads
	\begin{equation}
	H = \sum_{i=1}^{L} \left[ \lambda \left( \sigma^x_i - \sigma^z_{i-1} \sigma^x_i \sigma^z_{i+1} \right) + \mu\, \sigma^z_i + J\, \sigma^z_i \sigma^z_{i+1} \right]
	\equiv H_\lambda + H_z + H_{zz},
	\label{eq:H}
	\end{equation}
    where $\sigma^{x,y,z}_i$ are Pauli matrices defined on site $i$.

	

    \item \textbf{Reference:} \href{https://journals.aps.org/prb/abstract/10.1103/PhysRevB.101.024306}{Phys. Rev. B 101, 024306 (2020)}.
    \item \textbf{Analysis:}
    \begin{itemize}
        \item Restricted study to analytically known scarred eigenstates.\\ There are two towers of scars, both of length $L/2$ for OBC and $L/2 + 1$ for PBC.\\
         The first tower is described by:
	 \begin{equation}
	|S_n\rangle = \frac{1}{n! \sqrt{\mathcal{N}(L, n)}} \left(Q^\dagger\right)^n |\Omega\rangle, \quad n = 0,\hdots,L/2 - 1 \, (L/2)
	\end{equation}
	
	where $|\Omega\rangle = |0 \cdots 0\rangle$ and
	
	\begin{equation}
	\mathcal{N}(L, n) = \begin{cases}
	\binom{L - n - 1}{n} \quad &\text{OBC}\\
	\frac{L}{n} \binom{L - n - 1}{n - 1} \quad &\text{PBC}\\
	\end{cases}
	\end{equation}
	
	(\textcolor{red}{PBC not defined for $n=0$, even though it describes the first scar of the tower, i.e. state $|\Omega\rangle $}). 
	 
	 $Q^\dagger$ is 	the ladder operator, defined as:
	
	\begin{equation}
	Q^\dagger = \sum_{i = 1}^{L} (-1)^i P^0_{i-1} \, \sigma^+_i \, P^0_{i+1}, 
	\end{equation}
	
	where $\sigma_j^\pm = \frac{1}{2} (\sigma_j^x \pm i \sigma_j^y)$ and $P^0_i = \frac{1}{2} (1 - \sigma^z_i)$ is the local projector onto spin down.\\
        The second tower, related to the first one, is described as 
        \begin{equation}
	|S'_n\rangle = G |S_n\rangle = \frac{1}{n! \sqrt{\mathcal{N}(L, n)}} \left(Q'^\dagger\right)^n |\Omega'\rangle, \quad n = 0,\hdots,L/2 - 1 \, (L/2)
	\end{equation}
	
	\noindent where $G = \prod_{i=1}^L \sigma_i^x$ is a $\mathbb{Z}_2$ transformation that flips all spins, 
	$|\Omega'\rangle = |1 \cdots 1\rangle$, and
	
	\begin{equation}
	Q'^\dagger = G Q^\dagger G = \sum_{i = 1}^{L} (-1)^i P^1_{i-1} \, \sigma^-_i \, P^1_{i+1},
	\end{equation}
	
	\noindent with $P^1_i = \frac{1}{2}(1 + \sigma^z_i)$ the local projector onto spin up.\\
	Both scar towers are annihilated by $H_{\lambda}$, which breaks U(1): $H_{\lambda} |S_n\rangle = H_{\lambda} |S'_n\rangle = 0$.
        \item Bipartite entanglement entropy computed and compared with theoretical values.
        For the states $|S_n\rangle$, we have an exact formula for the bipartite entanglement entropy. Following the original reference, this is given by:
        
        \begin{equation}\label{aaa1}
        S^{|S_n\rangle }_A = - \sum_{k=0}^{K} \sum_{s=\pm} \lambda_{k,s} \ln \lambda_{k,s},
        \end{equation}
        
        where $K =\min{\left(n, \lfloor L_A/2 \rfloor\right)}$ and $\lambda_{k,s}$ are the eigenvalues of the RDM $\rho_A$ over subsystem A, which are 
        
        \begin{equation}\label{bbb1a}
        \lambda_{k,\pm} \left( L, L_A, n\right) = \frac{D_{1,k}m_{1,k} + D_{2,k}m_{2,k} \pm \sqrt{\left(D_{1,k}m_{1,k} - D_{2,k}m_{2,k}\right)^2 + 4 D_{1,k}D_{2,k}m^2_{2,k}}}{2\mathcal{N}}
        \end{equation}
        
        where
        
        \begin{align}
        &\mathcal{N} = \mathcal{N}\left(L, n\right)\\
        &D_{1,k} = D_{1,k}\left(L_A\right) = \mathcal{N}\left(L_A, k\right)\\
        &D_{2,k} = D_{2,k}\left(L_A\right) = \mathcal{N}\left(L_A - 1, k - 1\right)\\
        &m_{1,k} = m_{1,k}\left(L-L_A\right) = \mathcal{N}\left(L-L_A, n-k\right) + \mathcal{N}\left(L-L_A -1, n-k -1\right)\\
        &m_{2,k} = m_{2,k}\left(L-L_A\right) = \mathcal{N}\left(L-L_A, n-k\right)\\
        \end{align}
        
        For bipartite entanglement, $L_A = \frac{L}{2}$.\\
        By taking the asymptotic limit $L\rightarrow\infty$ of Eq. \eqref{bbb1a} one can show that the bipartite ee scales logarithmically with system size.
        The bipartite entanglement entropy was numerically computed and its consistency with Eq. \eqref{bbb1a} was verified, where we chose $|S_4\rangle$ for L=18 sites (with OBC!):
        
        \begin{equation}
        S^{|S_4\rangle }_A = - \sum_{k=0}^{4}\sum_{s=\pm} \lambda_{k,s} \ln \lambda_{k,s}
        \end{equation}
        
        which gives $S^{|S_4\rangle }_A \approx 1.22$, and it agrees with the numerical result.\\
        
        \item Ranks of adjacent 2-site, 3-site, and 4-site (adjacent) RDMs assessed, for both towers
        
        \begin{table}[H]
	\centering
	\begin{tabular}{|c|ccc|}
	\hline
	\textbf{$|S_n\rangle$} & \multicolumn{3}{c|}{\textbf{RDM Rank}} \\
	\cline{2-4}
	& \textbf{2-sites} & \textbf{3-sites} & \textbf{4-sites} \\
	\hline
	 n = 1 & 2 & 2 & 2 \\
	 n = 2 & 3 & 5 & 5 \\
	 n = 3 & 3 & 5 & 7 \\
	 n = 4 & 3 & 5 & 7 \\  
	 n = 5 & 3 & 5 & 7 \\
	 n = 6 & 3 & 5 & 7 \\
	 n = 7 & 3 & 5 & 7 \\
	 n = 8 & 3 & 5 & 6 \\
	\hline
	\end{tabular}
	\caption{RDM ranks for 2, 3, and 4 adjacent sites in the $|S_n\rangle$ scar tower for $L=18$.}
	\label{tab:ranks1}
	\end{table}

	 \begin{table}[H]
	\centering
	\begin{tabular}{|c|ccc|}
	\hline
	\textbf{$|S'_n\rangle$} & \multicolumn{3}{c|}{\textbf{RDM Rank}} \\
	\cline{2-4}
	& \textbf{2-sites} & \textbf{3-sites} & \textbf{4-sites} \\
	\hline
	 n = 1 & 2 & 2 & 2 \\
	 n = 2 & 3 & 5 & 5 \\
	 n = 3 & 3 & 5 & 7 \\
	 n = 4 & 3 & 5 & 7 \\  
	 n = 5 & 3 & 5 & 7 \\
	 n = 6 & 3 & 5 & 7 \\
	 n = 7 & 3 & 5 & 5 \\
	 n = 8 & 3 & 5 & 6 \\
	\hline
	\end{tabular}
	\caption{RDM ranks for 2, 3, and 4 adjacent sites in the $|S'_n\rangle$ scar tower for $L=18$.}
	\label{tab:ranks2}
	\end{table}
	
	The chain length $L$  dependence of the 2,3,4-sites RDM of the scar at the top of both tower was investigated too.
	
	\begin{figure}[H]
    \centering
    % First image
    \begin{subfigure}{0.45\textwidth}
        \includegraphics[width=\linewidth]{dw_scar_2.png}
        \caption{}
        \label{fig:image1}
    \end{subfigure}
    % Second image
    \begin{subfigure}{0.45\textwidth}
        \includegraphics[width=\linewidth]{dw_scar_3.png}
        \caption{}
        \label{fig:image2}
    \end{subfigure}    % Third image
    \begin{subfigure}{0.45\textwidth}
        \includegraphics[width=\linewidth]{dw_scar_4.png}
        \caption{}
        \label{fig:image3}
    \end{subfigure}

    % Common caption
    \caption{Innermost 2,3,4-sites RDM  rank (Fig. (a), (b), (c) respectively) of the top scar of the tower $|S_n\rangle$ as a function of increasing $L$.}
    \label{fig:dw_scars_tower}
\end{figure}

\begin{figure}[H]
    \centering
    % First image
    \begin{subfigure}{0.45\textwidth}
        \includegraphics[width=\linewidth]{dw_scar_2p.png}
        \caption{}
        \label{fig:image1p}
    \end{subfigure}
    % Second image
    \begin{subfigure}{0.45\textwidth}
        \includegraphics[width=\linewidth]{dw_scar_3p.png}
        \caption{}
        \label{fig:image2p}
    \end{subfigure}    % Third image
    \begin{subfigure}{0.45\textwidth}
        \includegraphics[width=\linewidth]{dw_scar_4p.png}
        \caption{}
        \label{fig:image3p}
    \end{subfigure}

    % Common caption
    \caption{Innermost 2,3,4-sites RDM  rank (Fig. (a), (b), (c) respectively) of the top scar of the tower $|S'_n\rangle$ as a function of increasing $L$.}
    \label{fig:dw_scars_towerp}
\end{figure}


\end{itemize}
\end{itemize}


\vspace{0.3cm}

\section*{3. Spin-1 XY Magnet Tower of Scars}
\begin{itemize}
    \item \textbf{Model:} Spin-1 XY model exhibiting a tower of scarred eigenstates.\\
    The Hamiltonian of the spin-1 XY model is:
	\begin{equation}\label{eqq}
	H = J \sum_{\langle i,j \rangle} \left( S^x_i S^x_j + S^y_i S^y_j \right) 
	+ h \sum_i S^z_i + D \sum_i \left(S^z_i\right)^2,
	\end{equation}
	
	where $S^\alpha_i$ ($\alpha = x, y, z$) are spin-1 operators defined on the sites of a 1-dimensional chain of length L, and $\langle i,j \rangle$ denotes nearest neighbors.  

	
    \item \textbf{Reference:} \href{https://journals.aps.org/prl/abstract/10.1103/PhysRevLett.123.147201}{Phys. Rev. Lett. 123, 147201 (2019)}.
    \item \textbf{Analysis:}
    \begin{itemize}
        \item Focused on exact scarred states with known analytic forms.\\ There are two towers of scars, both of length $L+1$.\\
    The first tower is described by:
	    \begin{equation}
	|S_n\rangle = \mathcal{N}(L, n) \left(J^\dagger\right)^n |\Omega\rangle, \quad n = 0,\hdots,L
	\end{equation}
	
	where $|\Omega\rangle = \bigotimes_i \left| m_i = -1 \right\rangle$ and $\mathcal{N}(L, n) = \sqrt{(L-n)!/n!L!}\,$. $J^\dagger$ is the ladder operator, defined as:
	
	\begin{equation}
	J^\dagger = \frac{1}{2}\sum_{i=1}^{L} (-1)^i \left(S^+_i\right)^2, 
	\end{equation}
	
	where $S_i^\pm = S_i^x \pm i S_i^y$.\\
     The second tower is given by 
     \begin{equation}
	|S'_n\rangle \propto \sum_{i_1 \neq \hdots \neq i_n} (-1)^{i_1 + \hdots + i_n} \left(S^+_{i_1}S^+_{i_1+1}\right)\hdots\left(S^+_{i_n}S^+_{i_n+1}\right)  |\Omega\rangle, \quad n = 0,\hdots,L
	\end{equation}
	Both scar towers are annihilated by the first term of \eqref{eqq}, which breaks what is known as Q-SU(2) symmetry (a special SU(2) symmetry that only scars have - see the reference \href{https://arxiv.org/pdf/2007.16207}{https://arxiv.org/pdf/2007.16207} for explanation): $J \sum_{\langle i,j \rangle} \left( S^x_i S^x_j + S^y_i S^y_j \right)  |S_n\rangle = J \sum_{\langle i,j \rangle} \left( S^x_i S^x_j + S^y_i S^y_j \right)  |S'_n\rangle = 0$.
        \item Bipartite entanglement entropy computed and compared with theoretical values.
        For the states $|S_n\rangle$, we have an exact formula for the bipartite entanglement entropy. Following the original reference, we can start from the reduced density matrix of a subsystem A:
        
        \begin{equation}\label{aaa1}
        \rho_A = \sum_{k=0}^{K} \lambda_k |k\rangle \langle k| \Rightarrow S^{|S_n\rangle }_A = - \sum_{k=0}^{K} \lambda_k \ln \lambda_k,
        \end{equation}
        
        where $K =\min{\left(n, L_A\right)}$ and
        
        \begin{equation}\label{bbb1}
        \lambda_k = \frac{\left( \begin{array}{c} L - L_A \\ n-k \end{array} \right)\left( \begin{array}{c} L_A \\ k \end{array} \right)}{\left( \begin{array}{c} L \\ n \end{array} \right)}
        \end{equation}
        
        $|k\rangle$ are the associated eigenvectors. For bipartite entanglement, $L_A = \frac{L}{2}$.\\
        By taking the asymptotic limit $L\rightarrow\infty$ of Eq. \eqref{aaa1} one can show that the bipartite ee scales logarithmically with system size:
        
        \begin{equation}
         S^{|S_n\rangle }_A = \frac{1}{2} \left\{ 1+ \ln \left[\pi \nu \left( 1-\nu\right) L\right] \right\}
        \end{equation}
        
        where $\nu=\frac{n}{L}$. This formula is consistent with sub-volume law behavior, typical of many-body scars.\\
        The bipartite entanglement entropy was numerically computed and its consistency with Eq. \eqref{aaa1} was verified, where we chose $|S_5\rangle$ for L=10 sites (with OBC!):
        
        \begin{equation}
        S^{|S_5\rangle }_A = - \sum_{k=0}^{5} \lambda_k \ln \lambda_k, \quad \lambda_k = \frac{\binom{5}{5-k}\binom{5}{k}}{\binom{10}{5}} =  \frac{\binom{5}{5-k}\binom{5}{k}}{252}
        \end{equation}
        
        which gives $S^{|S_5\rangle }_A \approx 1.23$, and it agrees with the numerical result.\\
        
        For generic $L$, $S^{\,|S_5\rangle}_A$ scales logarithmically with system size, i.e. $S^{\,|S_5\rangle}_A = \frac{1}{2} \ln \left(\frac{e \pi L}{8}\right)$. Because $S^{\,|S_5\rangle}_A$ is at the top of the tower - it has maximum possible value of entanglement entropy among the scars - we can assume that all scars have sub-volume scaling with system size, i.e. constant or at most logarithmic.
        
        \item Evaluated ranks of 2- to 4-site adjacent RDMs.
         \begin{table}[H]
	\centering
	\begin{tabular}{|c|ccc|}
	\hline
	\textbf{$|S_n\rangle$} & \multicolumn{3}{c|}{\textbf{RDM Rank}} \\
	\cline{2-4}
	& \textbf{2-sites} & \textbf{3-sites} & \textbf{4-sites}\\
	\hline
	 n = 0 & 1 & 1 & 1 \\
	 n = 1 & 2 & 2 & 2 \\
	 n = 2 & 3 & 3 & 3 \\
	 n = 3 & 3 & 4 & 4 \\
	 n = 4 & 3 & 4 & 5 \\  
	 n = 5 & 3 & 4 & 5 \\
	 n = 6 & 3 & 4 & 5 \\
	 n = 7 & 3 & 4 & 4 \\
	 n = 8 & 3 & 3 & 3 \\
	 n = 9 & 2 & 2 & 2 \\
	 n = 10 & 1 & 1 & 1 \\
	\hline
	\end{tabular}
	\caption{RDM ranks for 2, 3, and 4 adjacent sites in the $|S_n\rangle$ scar tower for $L=10$.}
	\label{tab:ranks21}
	\end{table}

	 \begin{table}[H]
	\centering
	\begin{tabular}{|c|ccc|}
	\hline
	\textbf{$|S'_n\rangle$} & \multicolumn{3}{c|}{\textbf{RDM Rank}} \\
	\cline{2-4}
	& \textbf{2-sites} & \textbf{3-sites} & \textbf{4-sites} \\
	\hline
          n = 0 & 1 & 1 & 1 \\
	 n = 1 & 4 & 4 & 4 \\
	 n = 2 & 6 & 8 & 8 \\
	 n = 3 & 7 & 11 & 12 \\
	 n = 4 & 7 & 12 & 15 \\  
	 n = 5 & 7 & 12 & 16 \\
	 n = 6 & 7 & 12 & 15 \\
	 n = 7 & 7 & 11 & 12 \\
	 n = 8 & 6 & 8 & 8 \\
	 n = 9 & 4 & 4 & 4 \\
	 n = 10 & 1 & 1 & 1 \\
	\hline
	\end{tabular}
	\caption{RDM ranks for 2, 3, and 4 adjacent sites in the $|S'_n\rangle$ scar tower for $L=10$.\\ \textbf{The scars $|S'_n\rangle$ are less rank-deficient than the $|S_n\rangle$ scars}.}
	\label{tab:ranks22}
	\end{table}
	
	\begin{figure}[H]
    \centering
    % First image
    \begin{subfigure}{0.45\textwidth}
        \includegraphics[width=\linewidth]{xy_scar_2.png}
        \caption{}
        \label{fig:image1xy}
    \end{subfigure}
    % Second image
    \begin{subfigure}{0.45\textwidth}
        \includegraphics[width=\linewidth]{xy_scar_3.png}
        \caption{}
        \label{fig:image2xy}
    \end{subfigure}    % Third image
    \begin{subfigure}{0.45\textwidth}
        \includegraphics[width=\linewidth]{xy_scar_4.png}
        \caption{}
        \label{fig:image3xy}
    \end{subfigure}

    % Common caption
    \caption{Innermost 2,3,4-sites RDM  rank (Fig. (a), (b), (c) respectively) of the top scar of the tower $|S_n\rangle$ as a function of increasing $L$.}
    \label{fig:xy_scars_tower}
\end{figure}

\begin{figure}[H]
    \centering
    % First image
    \begin{subfigure}{0.45\textwidth}
        \includegraphics[width=\linewidth]{xy_scar_2p.png}
        \caption{}
        \label{fig:image1pxy}
    \end{subfigure}
    % Second image
    \begin{subfigure}{0.45\textwidth}
        \includegraphics[width=\linewidth]{xy_scar_3p.png}
        \caption{}
        \label{fig:image2pxy}
    \end{subfigure}    % Third image
    \begin{subfigure}{0.45\textwidth}
        \includegraphics[width=\linewidth]{xy_scar_4p.png}
        \caption{}
        \label{fig:image3pxy}
    \end{subfigure}

    % Common caption
    \caption{Innermost 2,3,4-sites RDM  rank (Fig. (a), (b), (c) respectively) of the top scar of the tower $|S'_n\rangle$ as a function of increasing $L$.}
    \label{fig:xy_scars_towerp}
\end{figure}

    \end{itemize}
    \end{itemize}
       
\vspace{0.3cm}

\section*{4. Hubbard/Hirsch $\eta$-pairing Tower of Scars}
\begin{itemize}
  \item \textbf{Model:} 1D fermionic lattice model with on-site interactions and bond-charge coupling, known as the Hirsch model. The Hamiltonian is written as:
    \begin{equation}
    H = H_{\text{Hubbard}} + H_{\text{Hirsch}},
    \end{equation}
    where
    \begin{align}\label{eqh}
    H_{\text{Hubbard}} &= - t \sum_{\langle i, j \rangle, \sigma} \left( c^\dagger_{i\sigma} c_{j\sigma} + \text{h.c.} \right) + U \sum_i n_{i\uparrow} n_{i\downarrow} - \mu \sum_{i, \sigma} n_{i\sigma}, \\
    H_{\text{Hirsch}} &= X \sum_{\langle i, j \rangle, \sigma} \left( n_{i\bar{\sigma}} + n_{j\bar{\sigma}} \right) \left( c^\dagger_{i\sigma} c_{j\sigma} + \text{h.c.} \right),
    \end{align}
    and \( X \in \mathbb{R} \) is the Hirsch parameter that controls the strength of the bond-charge interaction. Here, \( c^\dagger_{i\sigma} \) creates a fermion of spin \( \sigma \in \{\uparrow, \downarrow\} \) at site \( i \), and \( n_{i\sigma} = c^\dagger_{i\sigma} c_{i\sigma} \). The sum \( \langle i, j \rangle \) runs over nearest neighbors, and \( \bar{\sigma} \) denotes the opposite spin of \( \sigma \). Without loss of generality, we can set $\mu=0$.

    \item \textbf{Reference:} \href{https://journals.aps.org/prb/abstract/10.1103/PhysRevB.102.075132}{Phys. Rev. B 102, 075132 (2020)}.

    \item \textbf{Analysis:}\\
    
    \begin{itemize} 
    
    \item There exists \textit{three} towers of scarred states, all of length $L+1$ ($L$ is the chain length).\\
    The first one is the known tower of $\eta$-pairing states, of the form:
    \begin{equation}
    |\eta_n\rangle = (\eta^\dag)^n |0\rangle, \quad n = 0,\hdots,L
    \end{equation}
    
    where
    
    \begin{equation}
    \eta^\dag = \sum_i (-1)^i c^\dagger_{i\uparrow} c^\dagger_{i\downarrow}
    \end{equation}
    
     is the \(\eta\)-pairing operator and \( |0\rangle \) is the vacuum state. These states are exact eigenstates of \( H \) with energy \( E_n = n\,U \), and they form a highest-weight tower under the SU(2) \(\eta\)-pairing symmetry. The scars are annihilated by  	     the SU(2) \(\eta\)-pairing symmetry-breaking term $H_{\text{Hirsch}}$: $H_{\text{Hirsch}}|S_n\rangle  = 0$.\\
    \vspace{1mm}\\
    The other two towers can be obtained by applying an appropriate ladder operator on the $\eta$-pairing states. We call these towers \(\tau\)-pairing states $|\tau_{m,n}\rangle$, and \(\varsigma\)-pairing states $|\varsigma_n\rangle$:
    
    \begin{align}
    &|\tau_{m,n}\rangle = (\tau^\dag)^m |\eta_n\rangle = (\tau^\dag)^m (\eta^\dag)^n |0\rangle \quad m=0,\hdots,L \\
    &|\varsigma_n\rangle = \varsigma^\dag |\eta_n\rangle = \varsigma^\dag (\eta^\dag)^n |0\rangle 
    \end{align}

    where
    
    \begin{align}
    &\tau^\dag = \sum_i (-1)^i c^\dagger_{i\uparrow} c^\dagger_{i+1\uparrow}\\
    &\varsigma^\dag = \sum_i (-1)^i \left( c^\dagger_{i\uparrow} c^\dagger_{i+1\downarrow} - c^\dagger_{i\downarrow} c^\dagger_{i+1\uparrow} \right)
    \end{align}
    
    $\tau^\dagger$ creates a nearest-neighbor triplet $\lvert\uparrow, \uparrow\rangle$ with momentum $\pi$, while $\varsigma^\dagger$ creates a nearest-neighbor singlet $\lvert\uparrow, \downarrow\rangle - \lvert\downarrow, \uparrow\rangle$ with momentum $\pi$. For PBC, we set $c^\dagger_{L+1\sigma} \equiv c^\dagger_{1\sigma}$. These states are again exact eigenstates of \( H \) with energy \( E_n = n\,U \). Additionally, $|\tau_{m,n}\rangle$ has total spin $S=m$.\\
Note that both $|\tau_{m,n}\rangle$, $|\varsigma_n\rangle$, unlike the $\eta$-pairing states, are in general not normalized, and therefore need to be normalized before looking at properties such as entanglement entropy or RDM rank.\\
Both towers \( |\tau_{m,n}\rangle \) and \( |\varsigma_n\rangle \) are annihilated by the following three kinetic terms, which preserve spin and particle number while commuting with the \(\eta\)-pairing operator:

\begin{equation}
H_{\text{kin},1} = \sum_{j,\sigma} |h, \sigma\rangle \langle \sigma, h|_{j,j+1} + h.c.
\end{equation}
\begin{equation}
H_{\text{kin},2} = \sum_{j} \left(|h, d\rangle + |d, h\rangle \right) \left(|\uparrow, \downarrow\rangle - |\downarrow, \uparrow\rangle \right)_{j,j+1}  + h.c.
\end{equation}
\begin{equation}
H_{\text{kin},3} = \sum_{j,\sigma} |d, \sigma\rangle \langle \sigma, d|_{j,j+1} + h.c.
\end{equation}

where, following the notation of the original paper, we have defined the states on site $j$:

\begin{align}
& |h\rangle_j = |n_{j,\uparrow} = 0, n_{j,\downarrow} = 0\rangle \\
& |d\rangle_j = |n_{j,\uparrow} = 1, n_{j,\downarrow} = 1\rangle \\
& |\uparrow\rangle_j = |n_{j,\uparrow} = 1, n_{j,\downarrow} = 0\rangle \\
& |\downarrow\rangle_j = |n_{j,\uparrow} = 0, n_{j,\downarrow} = 1\rangle
\end{align}

The original Hamiltonian \eqref{eqh} can be expressed in terms of these kinetic terms:

\begin{align}
    H_{\text{Hubbard}} &= - t \left(H_{\text{kin},1} + H_{\text{kin},2} - H_{\text{kin},3}\right) + U \sum_j |d\rangle \langle d|_j, \\
    H_{\text{Hirsch}} &= X \left(H_{\text{kin},2} - 2 H_{\text{kin},3}\right).
    \end{align}
    
To show that the towers \( |\tau_{m,n}\rangle \) and \( |\varsigma_n\rangle \) are annihilated by the three kinetic terms

\begin{equation}
H_{\text{kin},a} |\tau_{m,n}\rangle = H_{\text{kin},a} |\varsigma_n\rangle = 0, \quad \text{for } a = 1,2,3.
\end{equation}

one can first demonstrates that \( |\tau_{m,0}\rangle \) and \( |\varsigma_0\rangle \) are annihilated by them, then that 

\begin{equation}
[H_{\text{kin},a}, \eta^\dag ] |\tau_{m,n}\rangle = [H_{\text{kin},a}, \eta^\dag ] |\varsigma_n\rangle = 0, \quad \text{for } a = 1,2,3.
\end{equation}




        \item For the $\eta$-pairing states, we have an exact formula for the bipartite entanglement entropy. Following the original reference, we can start from the reduced density matrix of a subsystem A:
               
        \begin{equation}\label{aaa}
        \rho_A = \sum_{k=0}^{K} \lambda_k |k\rangle \langle k| \Rightarrow S^{|\eta_n\rangle }_A = - \sum_{k=0}^{K} \lambda_k \ln \lambda_k,
        \end{equation}
        
        where $K =\min{\left(n, L_A\right)}$ and
        
        \begin{align}\label{bbb}
        & \lambda_k = \frac{\left( \begin{array}{c} L - L_A \\ n-k \end{array} \right)\left( \begin{array}{c} L_A \\ k \end{array} \right)}{\left( \begin{array}{c} L \\ n \end{array} \right)}\\
        & |k\rangle = \sqrt{\frac{\left( L_A - k \right)!}{L_A!k!}} \left( \sum_{i=1}^{L_A} \left( -1\right)^i c^\dag_{i\uparrow} c^\dag_{i\downarrow} \right)^{\hspace{-0.1mm}k} |0\rangle_{\hspace{-0.1mm}A}
        \end{align}
         For bipartite entanglement, $L_A = \frac{L}{2}$.\\
        By taking the asymptotic limit $L\rightarrow\infty$ of Eq. \eqref{aaa} one can show that the bipartite ee scales logarithmically with system size:
        
        \begin{equation}
         S^{|\eta_n\rangle }_A = \frac{1}{2} \left\{ 1+ \ln \left[ \pi n \left( 1-\frac{n}{L}\right) \right] \right\}
        \end{equation}
        
        consistent with sub-volume law behavior, typical of many-body scars.\\
        The bipartite entanglement entropy was computed and its consistency with Eq. \eqref{aaa} was verified, where we chose $|\eta_6\rangle$ for L=12 sites:
        
        \begin{equation}
        S^{|\eta_6\rangle }_A = - \sum_{k=0}^{6} \lambda_k \ln \lambda_k, \quad \lambda_k = \frac{\binom{6}{6-k}\binom{6}{k}}{\binom{12}{6}} =  \frac{\binom{6}{6-k}\binom{6}{k}}{924}
        \end{equation}
        
        which gives $S^{|\eta_6\rangle }_A \approx 1.32$, and it agrees with the numerical result. The agreement was also verified for each one of the eigenvalues $\lambda_k$.
         
        \item Evaluated ranks of 2- to 4-site adjacent RDMs.\\
        Note that, only for the tower $|\eta_n\rangle$, we can theoretically check the rank values from Eq. \eqref{aaa},\eqref{bbb}: given that the RDM $\rho_A$ is in its Schmidt form, its rank is simply the number of non-zero eigenvalues $\lambda_k$. All of these theoretical ranks do in fact match their numerical counterparts.\\
        
         \begin{table}[H]
	\centering
	\begin{tabular}{|c|ccc|}
	\hline
	\textbf{$|\eta_n\rangle$} & \multicolumn{3}{c|}{\textbf{RDM Rank}} \\
	\cline{2-4}
	& \textbf{2-sites} & \textbf{3-sites} & \textbf{4-sites}\\
	\hline
	 n = 0 & 1 & 1 & 1 \\
	 n = 1 & 2 & 2 & 2 \\
	 n = 2 & 3 & 3 & 3 \\
	 n = 3 & 3 & 4 & 4 \\
	 n = 4 & 3 & 4 & 5 \\  
	 n = 5 & 3 & 4 & 5 \\
	 n = 6 & 3 & 4 & 5 \\
	 n = 7 & 3 & 4 & 5 \\
	 n = 8 & 3 & 4 & 5 \\
	 n = 9 & 3 & 4 & 4 \\
	 n = 10 & 3 & 3 & 3 \\
	 n = 11 & 2 & 2 & 2 \\
	 n = 12 & 1 & 1 & 1 \\
	\hline
	\end{tabular}
	\caption{RDM ranks for 2, 3, and 4 adjacent sites in the $|\eta_n\rangle$ scar tower for $L=12$.}
	\label{tab:ranks21}
	\end{table}

\begin{table}[H]
\centering

% Row 1: m = 1, 2, 3
\begin{tabular}{ccc}
% m = 1
\begin{tabular}{|c|c|c|c|}
	\hline
	\multicolumn{4}{|c|}{\textbf{RDM Rank ($m = 1$)}} \\
	\hline
	$n$ & 2-sites & 3-sites & 4-sites \\
	\hline
	0 & 4 & 4 & 4 \\
	1 & 7 & 8 & 8 \\
	2 & 8 & 11 & 12 \\
	3 & 8 & 12 & 15 \\
	4 & 8 & 12 & 16 \\
	5 & 8 & 12 & 16 \\
	6 & 8 & 12 & 16 \\
	7 & 8 & 12 & 15 \\
	8 & 8 & 11 & 12 \\
	9 & 7 & 8 & 8 \\
	10 & 4 & 4 & 4 \\
	11 & - & - & - \\
	12 & - & - & - \\
	\hline
\end{tabular}
&
% m = 2
\begin{tabular}{|c|c|c|c|}
	\hline
	\multicolumn{4}{|c|}{\textbf{RDM Rank ($m = 2$)}} \\
	\hline
	$n$ & 2-sites & 3-sites & 4-sites \\
	\hline
	0 & 4 & 6 & 8 \\
	1 & 7 & 11 & 15 \\
	2 & 8 & 14 & 20 \\
	3 & 8 & 15 & 23 \\
	4 & 8 & 15 & 24 \\
	5 & 8 & 15 & 23 \\
	6 & 8 & 14 & 20 \\
	7 & 7 & 11 & 15 \\
	8 & 4 & 6 & 8 \\
	9 & - & - & - \\
	10 & - & - & - \\
	11 & - & - & - \\
	12 & - & - & - \\
	\hline
\end{tabular}
&
% m = 3
\begin{tabular}{|c|c|c|c|}
	\hline
	\multicolumn{4}{|c|}{\textbf{RDM Rank ($m = 3$)}} \\
	\hline
	$n$ & 2-sites & 3-sites & 4-sites \\
	\hline
	0 & 4 & 6 & 8 \\
	1 & 7 & 11 & 15 \\
	2 & 8 & 14 & 20 \\
	3 & 8 & 15 & 22 \\
	4 & 8 & 14 & 20 \\
	5 & 7 & 11 & 15 \\
	6 & 4 & 6 & 8 \\
	7 & - & - & - \\
	8 & - & - & - \\
	9 & - & - & - \\
	10 & - & - & - \\
	11 & - & - & - \\
	12 & - & - & - \\
	\hline
\end{tabular}
\end{tabular}

\vspace{1em}

% Row 2: m = 4, 5, 6
\begin{tabular}{ccc}
% m = 4
\begin{tabular}{|c|c|c|c|}
	\hline
	\multicolumn{4}{|c|}{\textbf{RDM Rank ($m = 4$)}} \\
	\hline
	$n$ & 2-sites & 3-sites & 4-sites \\
	\hline
	0 & 4 & 6 & 8 \\
	1 & 7 & 11 & 14 \\
	2 & 8 & 13 & 16 \\
	3 & 7 & 11 & 14 \\
	4 & 4 & 6 & 8 \\
	5 & - & - & - \\
	6 & - & - & - \\
	7 & - & - & - \\
	8 & - & - & - \\
	9 & - & - & - \\
	10 & - & - & - \\
	11 & - & - & - \\
	12 & - & - & - \\
	\hline
\end{tabular}
&
% m = 5
\begin{tabular}{|c|c|c|c|}
	\hline
	\multicolumn{4}{|c|}{\textbf{RDM Rank ($m = 5$)}} \\
	\hline
	$n$ & 2-sites & 3-sites & 4-sites \\
	\hline
	0 & 4 & 4 & 4 \\
	1 & 6 & 6 & 6 \\
	2 & 4 & 4 & 4 \\
	3 & - & - & - \\
	4 & - & - & - \\
	5 & - & - & - \\
	6 & - & - & - \\
	7 & - & - & - \\
	8 & - & - & - \\
	9 & - & - & - \\
	10 & - & - & - \\
	11 & - & - & - \\
	12 & - & - & - \\
	\hline
\end{tabular}
&
% m = 6
\begin{tabular}{|c|c|c|c|}
	\hline
	\multicolumn{4}{|c|}{\textbf{RDM Rank ($m = 6$)}} \\
	\hline
	$n$ & 2-sites & 3-sites & 4-sites \\
	\hline
	0 & 1 & 1 & 1 \\
	1 & - & - & - \\
	2 & - & - & - \\
	3 & - & - & - \\
	4 & - & - & - \\
	5 & - & - & - \\
	6 & - & - & - \\
	7 & - & - & - \\
	8 & - & - & - \\
	9 & - & - & - \\
	10 & - & - & - \\
	11 & - & - & - \\
	12 & - & - & - \\
	\hline
\end{tabular}
\end{tabular}

\caption{RDM ranks for states $|\tau_{m,n}\rangle$ for various $m$ and $n$, across 2-, 3-, and 4-sites RDMs for $L=12$. The $-$ symbol indicates that the corresponding state is forbidden by the Pauli exclusion principle, and thus the density matrix is not defined in that case.}
\end{table}

	
\begin{table}[H]
	\centering
	\begin{tabular}{|c|ccc|}
	\hline
	\textbf{$|\varsigma_n\rangle$} & \multicolumn{3}{c|}{\textbf{RDM Rank}} \\
	\cline{2-4}
	& \textbf{2-sites} & \textbf{3-sites} & \textbf{4-sites}\\
	\hline
	 n = 0 & 6 & 6 & 6 \\
	 n = 1 & 11 & 12 & 12 \\
	 n = 2 & 12 & 17 & 18 \\
	 n = 3 & 12 & 18 & 23 \\
	 n = 4 & 12 & 18 & 24 \\  
	 n = 5 & 12 & 18 & 24 \\
	 n = 6 & 12 & 18 & 24 \\
	 n = 7 & 12 & 18 & 23 \\
	 n = 8 & 12 & 17 & 18 \\
	 n = 9 & 11 & 12 & 12 \\
	 n = 10 & 6 & 6 & 6 \\
	 n = 11 & - & - & - \\
	 n = 12 & - & - & - \\
	\hline
	\end{tabular}
	\caption{RDM ranks for 2, 3, and 4 adjacent sites in the $|\varsigma_n\rangle$ scar tower for $L=12$.}
	\label{tab:ranks22}
	\end{table}

    
    \begin{figure}[H]
    \centering
    % First image
    \begin{subfigure}{0.45\textwidth}
        \includegraphics[width=\linewidth]{hb_scar_2e.png}
        \caption{}
        \label{fig:image1h}
    \end{subfigure}
    % Second image
    \begin{subfigure}{0.45\textwidth}
        \includegraphics[width=\linewidth]{hb_scar_3e.png}
        \caption{}
        \label{fig:image2hb}
    \end{subfigure}    % Third image
    \begin{subfigure}{0.45\textwidth}
        \includegraphics[width=\linewidth]{hb_scar_4e.png}
        \caption{}
        \label{fig:image3hb}
    \end{subfigure}

    % Common caption
    \caption{Innermost 2,3,4-sites RDM  rank (Fig. (a), (b), (c) respectively) of the top scar ($n=L/2$) of the tower $|\eta_n\rangle$ as a function of increasing $L$.}
    \label{fig:hb_scars_tower}
\end{figure}

\begin{figure}[H]
    \centering
    % First image
    \begin{subfigure}{0.45\textwidth}
        \includegraphics[width=\linewidth]{hb_scar_2t.png}
        \caption{}
        \label{fig:image1h}
    \end{subfigure}
    % Second image
    \begin{subfigure}{0.45\textwidth}
        \includegraphics[width=\linewidth]{hb_scar_3t.png}
        \caption{}
        \label{fig:image2hb}
    \end{subfigure}    % Third image
    \begin{subfigure}{0.45\textwidth}
        \includegraphics[width=\linewidth]{hb_scar_4t.png}
        \caption{}
        \label{fig:image3hb}
    \end{subfigure}

    % Common caption
    \caption{Innermost 2,3,4-sites RDM rank (Fig. (a), (b), (c) respectively) of the top scar ($m=2, n=4,5,6$ - for $L=8,10,12$) of the tower $|\tau_{m,n}\rangle$ as a function of increasing $L$.}
    \label{fig:hb_scars_tower}
\end{figure}

\begin{figure}[H]
    \centering
    % First image
    \begin{subfigure}{0.45\textwidth}
        \includegraphics[width=\linewidth]{hb_scar_2s.png}
        \caption{}
        \label{fig:image1h}
    \end{subfigure}
    % Second image
    \begin{subfigure}{0.45\textwidth}
        \includegraphics[width=\linewidth]{hb_scar_3s.png}
        \caption{}
        \label{fig:image2hb}
    \end{subfigure}    % Third image
    \begin{subfigure}{0.45\textwidth}
        \includegraphics[width=\linewidth]{hb_scar_4s.png}
        \caption{}
        \label{fig:image3hb}
    \end{subfigure}

    % Common caption
    \caption{Innermost 2,3,4-sites RDM  rank (Fig. (a), (b), (c) respectively) of the top scar ($n=L/2 - 1$) of the tower $|\varsigma_n\rangle$ as a function of increasing $L$.}
    \label{fig:hb_scars_tower}
\end{figure}

    \end{itemize}
\end{itemize}


\vspace{0.5cm}
\pagebreak
\section*{Conclusion}

Across all models, the focus was placed exclusively on scarred states with known exact analytic expressions. The investigations confirm key entanglement features and low-rank structure of adjacent reduced density matrices, serving as diagnostics of scarred nonthermal behavior without requiring access to the full spectrum.

\end{document}
